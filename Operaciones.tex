\documentclass[11pt]{article}

\usepackage[utf8]{inputenc}     % Soporte para acentos y caracteres latinos
\usepackage[T1]{fontenc}
\usepackage{amsmath,amssymb}    % Paquetes para entornos y símbolos matemáticos
\usepackage{graphicx}           % (Opcional) Para incluir imágenes
\usepackage{hyperref}           % (Opcional) Para hipervínculos en el PDF

\begin{document}

\title{Operaciones Matemáticas del Código Operaciones.Java}
\author{Jose Maria Romero Davila}
\date{\today}
\maketitle

\section{Escala de Grises (Promedio)}

La escala de grises se calcula obteniendo el promedio de los tres canales de color (R, G y B). 
\[
\text{Gris} = \frac{R + G + B}{3}.
\]

\subsection{Mini Ejemplo}
Supongamos un píxel con valores \(R = 100\), \(G = 150\), \(B = 200\). Entonces:
\[
\text{Gris} = \frac{100 + 150 + 200}{3} = 150.
\]
El píxel resultante tendrá un valor de gris de 150.

\section{Imagen Binaria}

Para generar una imagen binaria, primero se calcula el promedio de los tres canales:
\[
\text{promedio} = \frac{R + G + B}{3}.
\]
Luego se compara con un umbral (\texttt{umbral}):
\[
\text{binario}(x, y) = 
\begin{cases}
255 & \text{si } \text{promedio} > \texttt{umbral},\\
0   & \text{en caso contrario}.
\end{cases}
\]

\subsection{Mini Ejemplo}
Sea un píxel con \(R = 50\), \(G = 100\), \(B = 150\) y un umbral de 100. Entonces:
\[
\text{promedio} = \frac{50 + 100 + 150}{3} = 100.
\]
Como \(\text{promedio} = 100 \leq \texttt{umbral}\), el píxel se establece en 0 (negro).

\section{Negativo}
Para generar el negativo de la imagen:
\[
R' = 255 - R, 
\quad G' = 255 - G, 
\quad B' = 255 - B.
\]

\subsection{Mini Ejemplo}
Para un píxel con \(R = 100\), \(G = 150\), \(B = 200\):
\[
R' = 255 - 100 = 155, \quad
G' = 255 - 150 = 105, \quad
B' = 255 - 200 = 55.
\]
El píxel resultante será \((155, 105, 55)\).

\section{Ecualización de Histograma (en Grises)}

\subsection{Conversión a Gris}
\[
\text{gris}(x, y) = \frac{R + G + B}{3}.
\]

\subsection{Cálculo de Histograma}
\[
\text{hist}[i] = \text{(número de píxeles con gris = } i), \quad i \in [0, 255].
\]

\subsection{Función de Distribución Acumulada (CDF)}
\[
\text{cdf}[0] = \text{hist}[0], \quad
\text{cdf}[i] = \text{cdf}[i - 1] + \text{hist}[i], \quad i = 1, 2, \ldots, 255.
\]
\[
\text{cdfMin} = \min \bigl\{\,\text{cdf}[i] > 0\bigr\}.
\]

\subsection{Look-Up Table (LUT)}
Sea \(\text{totalPixeles} = \text{ancho} \times \text{alto}\). Para cada nivel \(i\):
\[
\text{lut}[i] = \left\lfloor \frac{(\text{cdf}[i] - \text{cdfMin}) \times 255}{\text{totalPixeles} - \text{cdfMin}} \right\rceil,
\]
donde se hace \(\text{clamp}\) a \([0, 255]\) si es necesario.

\subsection{Aplicación de la LUT}
\[
\text{nuevoGris}(x, y) = \text{lut}\bigl[\text{gris}(x, y)\bigr].
\]

\subsection{Mini Ejemplo}
Supongamos una imagen de \(2 \times 2\) con valores de gris: \([50, 100, 150, 200]\).

1. **Histograma**:
   \[
   \text{hist}[50] = 1, \quad \text{hist}[100] = 1, \quad \text{hist}[150] = 1, \quad \text{hist}[200] = 1.
   \]

2. **CDF**:
   \[
   \text{cdf}[50] = 1, \quad \text{cdf}[100] = 2, \quad \text{cdf}[150] = 3, \quad \text{cdf}[200] = 4.
   \]
   \(\text{cdfMin} = 1\).

3. **LUT**:
   \[
   \text{lut}[50] = \left\lfloor \frac{(1 - 1) \times 255}{4 - 1} \right\rceil = 0,
   \]
   \[
   \text{lut}[100] = \left\lfloor \frac{(2 - 1) \times 255}{3} \right\rceil = 85,
   \]
   \[
   \text{lut}[150] = \left\lfloor \frac{(3 - 1) \times 255}{3} \right\rceil = 170,
   \]
   \[
   \text{lut}[200] = \left\lfloor \frac{(4 - 1) \times 255}{3} \right\rceil = 255.
   \]

4. **Aplicación de LUT**:
   Los nuevos valores de gris son: \([0, 85, 170, 255]\).

\section{Filtros de Suavizado}

\subsection{Filtro de Media}
Para una máscara de tamaño \(n \times n\):
\[
R_{\text{salida}} = \frac{ \sum \text{R}_{\text{vecinos}} }{n \times n}, \quad
G_{\text{salida}} = \frac{ \sum \text{G}_{\text{vecinos}} }{n \times n}, \quad
B_{\text{salida}} = \frac{ \sum \text{B}_{\text{vecinos}} }{n \times n}.
\]

\subsection{Mini Ejemplo}
Sea una ventana \(3 \times 3\) con valores de \(R\):
\[
\begin{bmatrix}
10 & 20 & 30\\
40 & 50 & 60\\
70 & 80 & 90
\end{bmatrix}.
\]
El valor central se calcula como:
\[
R_{\text{salida}} = \frac{10 + 20 + 30 + 40 + 50 + 60 + 70 + 80 + 90}{9} = 50.
\]

\section{Filtros de Detección de Bordes}

\subsection{Filtro Sobel}
\[
K_x = 
\begin{bmatrix}
-1 & 0 & 1\\
-2 & 0 & 2\\
-1 & 0 & 1
\end{bmatrix},
\quad
K_y =
\begin{bmatrix}
-1 & -2 & -1\\
\;\,0 & \;\,0 & \;\,0\\
\;\,1 & \;\,2 & \;\,1
\end{bmatrix}.
\]
La magnitud resultante se obtiene con:
\[
R' = \sqrt{( \text{sumXr} )^2 + ( \text{sumYr} )^2}.
\]

\subsection{Mini Ejemplo}
Para una ventana \(3 \times 3\) con valores de \(R\):
\[
\begin{bmatrix}
10 & 20 & 30\\
40 & 50 & 60\\
70 & 80 & 90
\end{bmatrix},
\]
las sumas son:
\[
\text{sumXr} = (-1 \times 10) + (1 \times 30) + (-2 \times 40) + (2 \times 60) + (-1 \times 70) + (1 \times 90) = 80,
\]
\[
\text{sumYr} = (-1 \times 10) + (-2 \times 20) + (-1 \times 30) + (1 \times 70) + (2 \times 80) + (1 \times 90) = 160.
\]
La magnitud es:
\[
R' = \sqrt{80^2 + 160^2} = \sqrt{6400 + 25600} = \sqrt{32000} \approx 178.89.
\]

\section{Clampeo (\textit{Clamp})}

Los valores finales de cada canal se ajustan al rango \([0, 255]\):
\[
\text{val} = 
\begin{cases}
0 & \text{si } \text{val} < 0,\\
255 & \text{si } \text{val} > 255,\\
\text{val} & \text{en caso contrario}.
\end{cases}
\]

\subsection{Mini Ejemplo}
Si \(\text{val} = 300\), entonces \(\text{val} = 255\). Si \(\text{val} = -10\), entonces \(\text{val} = 0\).

\end{document}
