\documentclass[12pt]{article}
\usepackage[utf8]{inputenc}
\usepackage[spanish]{babel}
\usepackage{amsmath, amssymb}
\usepackage{graphicx}
\usepackage{geometry}
\usepackage{xcolor}
\usepackage{hyperref}
\geometry{margin=2.5cm}

\title{Procesamiento Digital de Imágenes 2D en Java}
\author{Autor: Jose Maria Romero Davila}
\date{\today}

\begin{document}

\maketitle
\tableofcontents
\newpage

\section{Introducción}

Este proyecto implementa un software de procesamiento digital de imágenes 2D en Java con una interfaz gráfica basada en Swing. Se ofrecen transformaciones básicas, filtros espaciales, operaciones morfológicas, histogramas y herramientas de edición como deshacer/rehacer.

\section{Fórmulas Matemáticas Utilizadas}

Esta sección detalla las operaciones implementadas en términos matemáticos y explica con ejemplos cómo se aplican a los píxeles de una imagen.

\subsection{Escala de Grises}

Convierte un píxel RGB a una sola intensidad promedio:

\[
G(x, y) = \frac{R(x, y) + G(x, y) + B(x, y)}{3}
\]

\textbf{Ejemplo:}  
Para un píxel (200, 150, 100), su gris será:  
$G = (200 + 150 + 100)/3 = 150$  
Resultado: (150, 150, 150)

\subsection{Imagen Binaria}

Aplica umbralización para obtener una imagen en blanco y negro:

\[
B(x, y) =
\begin{cases}
255 & \text{si } G(x, y) > T \\
0   & \text{si } G(x, y) \leq T
\end{cases}
\]

\textbf{Ejemplo:}  
Con umbral $T=128$:  
\begin{itemize}
  \item $G=150 \Rightarrow B=255$ (blanco)
  \item $G=80 \Rightarrow B=0$ (negro)
\end{itemize}

\subsection{Negativo de Imagen}

Invierte los canales de color:

\[
I'(x, y) = 255 - I(x, y)
\]

\textbf{Ejemplo:}  
RGB = (100, 180, 200) $\Rightarrow$ (155, 75, 55)

\subsection{Filtro de Media}

Reduce el ruido suavizando la imagen mediante un promedio:

\[
M(x, y) = \frac{1}{n^2} \sum_{i=-k}^{k} \sum_{j=-k}^{k} I(x+i, y+j)
\]
donde $k = \frac{n-1}{2}$

\textbf{Ejemplo:}  
Con máscara $3\times3$, se suman 9 valores y se divide por 9.

\subsection{Filtro de Mediana}

Reemplaza el valor del píxel por la mediana de su vecindario:

\[
M(x, y) = \text{mediana} \left\{ I(x+i, y+j) \mid i,j \in [-k,k] \right\}
\]

\textbf{Ejemplo:}  
Valores = \{20, 30, 40, 50, 60, 70, 80, 90, 100\}  
$\Rightarrow$ mediana = 60

\subsection{Detección de Bordes}

\subsubsection{Sobel}

\[
G_x =
\begin{bmatrix}
-1 & 0 & 1\\
-2 & 0 & 2\\
-1 & 0 & 1
\end{bmatrix}, \quad
G_y =
\begin{bmatrix}
-1 & -2 & -1\\
0 & 0 & 0\\
1 & 2 & 1
\end{bmatrix}
\]

\[
G(x,y) = \sqrt{(I * G_x)^2 + (I * G_y)^2}
\]

\textbf{Ejemplo:}  
Detecta bordes verticales y horizontales simultáneamente.

\subsubsection{Prewitt}

\[
P_x =
\begin{bmatrix}
-1 & 0 & 1\\
-1 & 0 & 1\\
-1 & 0 & 1
\end{bmatrix}, \quad
P_y =
\begin{bmatrix}
-1 & -1 & -1\\
0 & 0 & 0\\
1 & 1 & 1
\end{bmatrix}
\]

\textbf{Ejemplo:}  
Similar a Sobel pero más simple, con menos sensibilidad.

\subsubsection{Laplaciano}

\[
L =
\begin{bmatrix}
0 & -1 & 0\\
-1 & 4 & -1\\
0 & -1 & 0
\end{bmatrix}
\]

\[
G(x,y) = I * L
\]

\textbf{Ejemplo:}  
Detecta contornos sin orientación específica.

\subsection{Morfología Matemática}

\subsubsection{Erosión}

\[
I \ominus B = \min \left\{ I(x+i, y+j) \mid (i,j) \in B \right\}
\]

\textbf{Ejemplo:}  
Un punto blanco aislado se elimina.

\subsubsection{Dilatación}

\[
I \oplus B = \max \left\{ I(x+i, y+j) \mid (i,j) \in B \right\}
\]

\textbf{Ejemplo:}  
Aumenta áreas blancas, rellenando huecos.

\subsubsection{Apertura}

\[
I \circ B = (I \ominus B) \oplus B
\]

\textbf{Ejemplo:}  
Elimina ruido pequeño.

\subsubsection{Cierre}

\[
I \bullet B = (I \oplus B) \ominus B
\]

\textbf{Ejemplo:}  
Rellena pequeños agujeros.

\subsubsection{Esqueletonización}

\[
S = \bigcup_n \left[ I_n - ((I_n \ominus B) \oplus B) \right]
\]

\textbf{Ejemplo:}  
Reduce formas binarizadas al "hueso" o trazo principal.

\subsection{Histograma de Intensidad}

Cuenta cuántos píxeles tienen un valor de gris específico:

\[
H(g) = \#\left\{(x, y) \mid G(x, y) = g \right\}, \quad g \in [0, 255]
\]

\textbf{Ejemplo:}  
$H(128) = 250$ significa que hay 250 píxeles con intensidad 128.

\section{Conclusión}

Este documento ha descrito los fundamentos matemáticos de cada operación de procesamiento implementada en el software. La aplicación ofrece herramientas completas para experimentar con filtros clásicos y morfología, visualizando resultados de forma interactiva.

\end{document}
