\documentclass[12pt]{article}
\usepackage[utf8]{inputenc}
\usepackage[spanish]{babel}
\usepackage{amsmath, amssymb}
\usepackage{graphicx}
\usepackage{hyperref}
\usepackage{xcolor}
\usepackage{geometry}
\geometry{margin=2.5cm}

\title{Procesamiento de Imágenes 2D en Java}
\author{Autor: Jose Maria Romero Davila}
\date{\today}

\begin{document}

\maketitle
\tableofcontents
\newpage

\section{Introducción}

El procesamiento de imágenes es una disciplina fundamental en el campo de la visión por computadora y la gráfica computacional. En esta actividad, se ha desarrollado un software en Java con una interfaz gráfica de usuario (GUI) para la manipulación y análisis de imágenes en 2D. 

El objetivo principal es implementar diversas técnicas de procesamiento digital de imágenes, como conversión a escala de grises, binarización, aplicación de filtros espaciales y morfológicos, así como análisis en el dominio de la frecuencia.

El desarrollo del software se llevó a cabo en GitHub, donde se encuentra disponible el código fuente junto con documentación detallada sobre su funcionamiento. 

\textbf{Repositorio en GitHub:} \url{https://github.com/JoseDavila24/ImageProcessor2D}

\section{Resumen del Proyecto}

Este proyecto es una aplicación de escritorio desarrollada en Java para realizar operaciones de procesamiento digital de imágenes bidimensionales. Utiliza una interfaz gráfica implementada con \texttt{Swing} para permitir al usuario:

\begin{itemize}
    \item Abrir, visualizar y guardar imágenes.
    \item Aplicar una variedad de filtros y transformaciones.
    \item Usar herramientas como histograma, undo/redo, redimensionado dinámico.
\end{itemize}

La aplicación está orientada tanto a fines educativos como prácticos, con un enfoque modular y extensible.

\section{Características Principales}

\begin{itemize}
    \item Interfaz gráfica moderna con menús interactivos.
    \item Transformaciones:
    \begin{itemize}
        \item Escala de grises paralelizada
        \item Imagen binaria con umbral configurable
        \item Negativo
    \end{itemize}
    \item Filtros espaciales:
    \begin{itemize}
        \item Suavizado: Media, Mediana
        \item Bordes: Sobel, Prewitt, Laplaciano
    \end{itemize}
    \item Morfología matemática:
    \begin{itemize}
        \item Erosión, Dilatación
        \item Apertura, Cierre
        \item Esqueletonización iterativa
    \end{itemize}
    \item Filtros simulados de frecuencia:
    \begin{itemize}
        \item Paso bajo, alto y de banda
    \end{itemize}
    \item Histograma visual con etiquetas y escala
    \item Pilas de deshacer y rehacer para edición no destructiva
\end{itemize}

\section{Estructura del Proyecto}

\subsection*{1. Menu.java}
Clase principal de la interfaz. Gestiona eventos de menú, imagen actual, filtros, redimensionado y pilas de undo/redo.

\subsection*{2. Operaciones.java}
Contiene métodos estáticos que aplican los filtros. Agrupados en categorías: básicos, suavizado, bordes, morfología, frecuencia.

\subsection*{3. HistogramaPanel.java}
Dibuja el histograma de niveles de gris con etiquetas de intensidad y frecuencia, usando gráficos en 2D.

\subsection*{4. Main.java}
Inicializa la aplicación invocando la ventana principal desde el hilo de evento de Swing.

\section{Nuevas Funcionalidades Añadidas}

\subsection{Histograma Visual Mejorado}

Se ha implementado un panel de histograma con:
\begin{itemize}
    \item Barras para cada nivel de gris (0 a 255).
    \item Escala de frecuencia en eje Y.
    \item Etiquetas visibles en ambos ejes.
\end{itemize}

\subsection{Paralelización en Filtros}

Ejemplo aplicado en:
\begin{itemize}
    \item Escala de grises usando \texttt{IntStream.parallel()}.
\end{itemize}
Esto mejora el rendimiento en imágenes grandes dividiendo el procesamiento por hilos.

\subsection{Filtros Morfológicos Completos}

Incluye los siguientes filtros:

\begin{itemize}
    \item \textbf{Apertura:} Erosión seguida de dilatación. Elimina ruido fino.
    \item \textbf{Cierre:} Dilatación seguida de erosión. Rellena huecos pequeños.
    \item \textbf{Esqueletonización:} Reducción iterativa hasta el trazo central de figuras.
\end{itemize}

\section{Compilación y Ejecución}

\subsection*{Opción A: Ejecutar JAR}
\begin{enumerate}
    \item Descargar \texttt{ImageProcessor2D.jar}
    \item Desde consola, ejecutar:
    \begin{verbatim}
    java -jar ImageProcessor2D.jar
    \end{verbatim}
    \item O doble clic si está asociado con Java.
\end{enumerate}

\subsection*{Opción B: Compilar manualmente}
\begin{verbatim}
git clone https://github.com/JoseDavila24/ImageProcessor2D.git
cd ImageProcessor2D
javac Menu.java Operaciones.java HistogramaPanel.java Main.java
java Main
\end{verbatim}

\section{Instrucciones de Uso}

\begin{itemize}
    \item \textbf{Abrir imagen:} Archivo $\rightarrow$ Abrir imagen
    \item \textbf{Aplicar filtro:} Menús Imagen / Filtros
    \item \textbf{Deshacer/Rehacer:} Edición $\rightarrow$ Deshacer/Rehacer
    \item \textbf{Guardar imagen:} Archivo $\rightarrow$ Guardar / Guardar como
    \item \textbf{Histograma:} Imagen $\rightarrow$ Histograma
\end{itemize}

\section{Requisitos}

\begin{itemize}
    \item Java 8 o posterior
    \item Git (opcional)
    \item IDE (opcional, recomendado: IntelliJ IDEA, Eclipse)
\end{itemize}

\section{Licencia}

Este proyecto está bajo la Licencia MIT. Puedes reutilizar, modificar o distribuir el software con atribución.

\section{Referencias}

\begin{itemize}
    \item GitHub - jercyae7/editor-de-imagenes: \url{https://github.com/jercyae7/editor-de-imagenes}
    \item Torres, A. D. (1996). Procesamiento digital de imágenes. Perfiles Educativos, 72.
    \item Redalyc: \url{https://www.redalyc.org/pdf/132/13207206.pdf}
\end{itemize}

\section{Conclusión}

El uso de GitHub en este proyecto resultó fundamental, no solo para gestionar el código y la documentación, sino también para acostumbrarme a un entorno ampliamente utilizado en el desarrollo de software.

Como estudiante de Ingeniería en Sistemas Computacionales, familiarizarme con herramientas de control de versiones es esencial, ya que facilita la colaboración, el seguimiento de cambios y la organización de proyectos.

Esta experiencia me ha permitido mejorar mis habilidades en el manejo de repositorios, lo que será de gran utilidad en futuros desarrollos dentro del ámbito de TI.

\end{document}
